\section{Experimental results and discussions}
\label{sec:results}
\subsection{Experiment setup}
The experiment is set up on a Linux server with 2 Xeon E5-2698v2
processors and Nvidia Titan X RTX GPU. The applied VAE-based model is built with the {\it PyTorch} package. The linear
programming-based power grid fixing part is implemented in Python. The
test cases are randomly generated following the IBM-format, based on
the topology we extract from the real design. The detailed set up procedure
starts from synthesizing the power grid of the Cortex-M0 DesignStart 
processor, which implements ARMv6-M 32-bit architecture and is routed and placed
using IC Compiler with 32/28nm Generic Library from Synopsys. The Cortex power grid
consists of two layers and one thousand interconnect trees.

The IC Compiler output power grid information is sent to the power grid file parser.
The file parser reorganize the information includes structure, wire layer, wire length, 
wire resistance values, node location, voltage and current source, etc. The output from power grid
file parser is consistent with the IBM power grid benchmarks \cite{Nassif:ASPDAC'08}. We generate 
large amount of IBM-format power grid networks to ensure different workloads can be tested and verified.\\

We have three different topology designs. To train the model, each topology needs its own dataset containing more than 10000 pairs of samples  (workloads and aging time, EM-induced IR drop). The maximum allowable IR drop is set to $5\% \ V_{dd}$ and the target EM lifetime $T$ is 10 years. 







%\subsection{The performance and CPU costs comparisons}
\subsection{The result and performance}
We implemented the excessive linear programming method in~\cite{Sukharev:2019pg} and our proposed method for comparison. 

\begin{table}[!h]
	\begin{center} 
		\caption{Prediction results of VAE model and GAN model on different designs}
		\label{table: Model_RMSE_Compare}
		\center
			\begin{tabular}{| c| c | c | c | }
				\hline 
				{} &{} &\multicolumn{2}{c|} {RMSE (mV)}  \\
				\cline{3-4}
				\multirow{-2}{*}{Circuit}   &\multirow{-2}{*}{\# nodes} &{ GAN}  &{ VAE}   \\ \hline 
				\hline 
				Design1  &1024      &5.697 	& 3.144 	 \\ \hline
				Design2  &4096      &6.100	&  3.519	\\ \hline
				Design3  &16384   &3.922	        &  3.462	\\ \hline			
			\end{tabular}
	\end{center}
	\vspace{-0.1in}
\end{table}


\begin{table}[h]
	\begin{center} %\footnotesize
		\caption{ Comparison of the proposed VAE-accelerated SLP optimization method against the existing method, need update data \textcolor{red}{need update data}}
		\label{table:results}
		\center
		\resizebox{0.48\textwidth}{!}{
			\begin{tabular}{| c | c | c | c | c | c | c | }
				\hline 	
				{} &{} &\multicolumn{2}{c|} {Existing SLP~\cite{Sukharev:2019pg}}  & \multicolumn{2}{c|} {VAE-accelerated SLP} &{} \\ 
				\cline{3-6} \multirow{-2}{*}{Circuit} & \multirow{-2}{*}{PG Size} & {Area Increase} & {Time(s)}   &{Area Increase} & {Time (s)} & \multirow{-2}{*}{Speed up}  \\  [4pt]
				\hline 
				\hline 
				Design1-PG1  &1024     &1.61\%	&41.8      & 0.98\%	&2.5    &16.79		\\ [5pt]
				\hline
				Design2-PG2  &4096     & 1.55\%	& 165.8      & 1.79\%	        & 7.6   &21.82s		\\	[5pt]
				\hline
				Design2-PG3  &4096     &\%	&      &\%	&   &	\\	[5pt]
				\hline
				Design3-PG4  &16384   &\% 	&       &\%	&  &		\\	[5pt]
				\hline
				Design3-PG5  &16384   &\% 	&      &\%	&  & \\	[5pt]
				\hline
			\end{tabular}
		}
	\end{center}
	\vspace{-0.1in}
\end{table}
      
      The table \ref{table:results} shows the comparison of our
      proposed method against the existing SLP-based
      method~\cite{Sukharev:2019pg} and DNN-based conjugate gradient
      method~\cite{HanLiu:TCAD'22-23} for the power grid benchmarks in
      Table~\ref{table:pre_results}. The last two columns indicate
      the proposed method has the speedup over ~\cite{Sukharev:2019pg}
      and speedup over ~\cite{HanLiu:TCAD'22-23} respectively.
      As we can see that both SLP-based
      methods can lead to similar performance in terms of area
      increases from the optimization process. In some cases the
      conjugate gradient method leads to a smaller area increase,
      that's because we set the conjugate gradient method to only
      widen the interconnect branch, not the entire interconnect tree
      which the branch belong to. In irregular power grid, an
      interconnect tree usually consists of very few branches, and the
      area increase difference is expected to be smaller than our mesh grid type design.
      On the other hand, we see that our DNN-accelerated SLP method indeed
      outperform the conventional SLP-based method. As the sizes of
      the power grid increase, we see more speedup.
      








