\documentclass[journal]{IEEEtran}
% \usepackage{booktabs} % For formal tables
% \usepackage{times}
% \renewcommand{\rmdefault}{ptm}
\usepackage{multirow}
% \usepackage{array}
\usepackage{amsmath}
\usepackage{subfigure}
% \usepackage{bm}
% \usepackage{amsfonts}
\usepackage{epsfig}
\usepackage{psfrag}
% \usepackage{epstopdf}
%\usepackage{cite}
%\usepackage{url}
\newcounter{numberlistc}
\usepackage{algorithm}
\usepackage{algorithmic}
\usepackage{graphicx}
\usepackage{amssymb}
\usepackage{amsthm}
\usepackage{subfigure}
\usepackage{color}
\usepackage{spverbatim}
\usepackage{setspace}
\usepackage{caption}
%\usepackage{subcaption}

\newenvironment{numberlist}
    {   \setcounter{numberlistc}{0}
        \begin{list}{\arabic{numberlistc}.}
        {\usecounter{numberlistc}
        \setlength{\parsep}{0pt}
        \setlength{\topsep}{3pt}
        \setlength{\itemsep}{0pt}}
        }{ \end{list} }
\newcounter{itemlistc}
\newcounter{enumlistc}
%\renewcommand{\theromanlistc}{(\roman{romanlistc})} %for ref use
%\renewcommand{\thealphlistc}{(\alph{alphlistc})}    %for ref use
\newenvironment{itemlist}
    {   \setcounter{itemlistc}{0}
    \begin{list}{$\bullet$}
        {\usecounter{itemlistc}
        \setlength{\parsep}{0pt}
        \setlength{\topsep}{3pt}
        \setlength{\itemsep}{0pt}}
        }{ \end{list} } 

\newcommand{\rev}[1]{\textcolor{red}{#1}}

\renewcommand{\baselinestretch}{0.96}
%\renewcommand{\baselinestretch}{1}
% \newenvironment{packedenum}{
% \begin{enumerate}
% \setlength{\itemsep}{1pt}
% \setlength{\parskip}{0pt}
% \setlength{\parsep}{0pt}
% }{\end{enumerate}}

%\usepackage{setspace}
%\DeclareMathOperator{\rank}{rank}


%\renewcommand{\baselinestretch}{0.895}
%\IEEEoverridecommandlockouts
%\usepackage[font=small, labelfont=bf]{caption}
%\usepackage{caption}


%\renewcommand{\algorithmicrequire}{\textbf{Input:}}
%\renewcommand{\algorithmicensure}{\textbf{Output:}}
%\renewcommand{\algorithmiccomment}[1]{~~~~\textcolor{BrickRed}{// \textit{#1}}}


\begin{document}
\title{GridVAE: Fast Power Grid EM-Aware IR Drop Prediction and Fixing
  Accelerated by Variational AutoEncoder}


\author{
Yibo Liu~~\IEEEmembership{Student Member,~IEEE}, Sheldon X.-D. Tan~\IEEEmembership{Senior Member,~IEEE}
~\thanks{
 \indent  Yibo Liu, and Sheldon X.-D Tan are with the Department of
 Electrical and Computer Engineering,  University of California,
 Riverside, CA 92521 USA
  \newline \indent
This work is supported in part by NSF grants under
No.CCF-2305437.
}
}

\maketitle
  
\begin{abstract}
  % Electromigration (EM) is the top failure mechanism for copper-based
  % interconnected for today's and future nanometer chip technologies.
  % The on-chip power grids have to be properly design to ensure the EM
  % life time. Most of the existing approach is based on the sensitivity
  % based optimizations and one critical issue is the high computational
  % costs of full-chip EM-aware IR drop analysis to compute the
  % sensitivity of the fixing target with respect to the wire
  % geometry. In this paper, we propose a fast full-chip power grid
  % EM-aware IR drop prediction and fixing framework. First, we adopt
  % the variational autoencoder (VAE)-based model, called {\it GridVAE},
  % to predict the EM-aware IR drops, which are more accurate and thus
  % training data-efficient than existing generative adversarial network
  % (GAN) based approaches. On top of this, we apply a sequence of
  % linear programming based optimization framework to size the wires by
  % leveraging the auto differential feature of the deep neural networks
  % to compute the sensitivity information very efficiently.  The
  % numerical results show that VAE is better than recently proposed
  % recently proposed GAN-based model with 40$\%$ RMSE reduction.
  % Further on the power grid EM-aware IR drop fixing, we show the
  % proposed VAE-accelerated method can lead up to 80X (at least one
  % order of magnitude) speedup over the conventional SLP-based method
  % on synthesized power grid benchmarks from ARM Cortex-M0 processor
  % design.


  Electromigration (EM) remains the primary failure mechanism for
  copper-based interconnects in today's and future nanometer chip
  technologies. To ensure the longevity of on-chip power grids,
  effective EM-aware IR drop analysis is crucial. However, the
  existing power grid optimization approaches suffer high
  computational costs from the full-chip EM-aware IR drop analysis and
  sensitivity computation.  This paper proposes a novel and efficient
  framework for full-chip power grid EM-aware IR drop prediction and
  fixing framework.  We developed a conditional VAE-based framework,
  named {\it GridVAE}, for fast and accurate EM-aware IR drop
  prediction and full-chip power grid fixing. Compared to the
  state-of-the-art generative adversarial network (GAN)-based methods,
  our GridVAE model offers a remarkable 40\% reduction in prediction
  RMSE on synthesized power grid benchmarks from ARM Cortex-M0
  processor design.  Building on the accurate EM-aware IR drop
  predictions and fast acquired sensitivities, we apply the sequence
  of linear programming-based optimizations to efficiently size the
  wires.
  %significantly reducing computation time for sensitivity information computation. 
  Our proposed GridVAE method achieves up to an 140X speedup (at least one order of magnitude) compared to conventional SLP-based methods for power grid EM-aware IR drop fixing. 
\end{abstract}
 
\input intro.tex

\input related.tex

\input strategy.tex

\input results.tex

\section{Conclusion}
\label{sec:conclusion}
% In this paper, we proposed a variational autoencoder (VAE) based
% full-chip EM-aware IR drop estimation model, called {\it GridVAE}, for
% on-chip power grid networks. The new deep neural network models allow
% fast sensitivity computations of target cost function with respect to
% the wire geometries. This enable accelerated power grid fixing based
% on a recently proposed linear programming based optimization
% framework.  The numerical results show that {\it GridVAE} is better
% than recently proposed recently proposed GAN-based model with 40$\%$
% RMSE reduction.  Furthermore, we show the proposed {\it
%   GridVAE}-accelerated method can lead up to 80X (at least one order
% of magnitude) speedup over the conventional SLP-based method on
% synthesized power grid benchmarks from ARM Cortex-M0 processor design.


In conclusion, our work introduced a novel full-chip EM-aware IR drop estimation model, referred to as {\it GridVAE},  tailored specifically for on-chip power grid EM-aware IR drop prediction and rapid sensitivity computations, and accelerated the power grid fixing through a recently proposed linear programming-based optimization framework. 
The  {\it GridVAE} reduced 40\% RMSE over the recently proposed state-of-the-art GAN-based model on full-chip EM-aware IR drop prediction. 
Additionally, our GridVAE-accelerated method demonstrates an impressive up to 140X speedup (at least one order of
magnitude) compared to conventional SLP-based approaches when applied
to synthesized power grid benchmarks from ARM Cortex-M0 processor
design. These findings underscore the potential of {\it GridVAE} as a
promising tool for enhancing power grid optimization in modern
nanometer-scale chip technologies, providing valuable insights into
efficient and accurate solutions for addressing EM-related challenges
in VLSI design.

% \clearpage 
\bibliographystyle{IEEEtran}
\bibliography{../../bib/simulation,../../bib/modeling,../../bib/reduction,../../bib/misc,../../bib/mscad_pub,../../bib/reliability,../../bib/interconnect,../../bib/thermal_power,../../bib/machine_learning,../../bib/physical,../../bib/neural_network}
%\bibliography{bibfile}
%\input change_note.tex
% \input {bio.tex}

\end{document}
