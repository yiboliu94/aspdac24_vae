\documentclass[journal]{IEEEtran}
% \usepackage{booktabs} % For formal tables
% \usepackage{times}
% \renewcommand{\rmdefault}{ptm}
\usepackage{multirow}
% \usepackage{array}
\usepackage{amsmath}
\usepackage{subfigure}
% \usepackage{bm}
% \usepackage{amsfonts}
\usepackage{epsfig}
\usepackage{psfrag}
% \usepackage{epstopdf}
%\usepackage{cite}
%\usepackage{url}
\newcounter{numberlistc}
\usepackage{algorithm}
\usepackage{algorithmic}
\usepackage{graphicx}
\usepackage{amssymb}
\usepackage{amsthm}
\usepackage{subfigure}
\usepackage{color}
\usepackage{spverbatim}
\usepackage{setspace}
\usepackage{caption}
%\usepackage{subcaption}

\newenvironment{numberlist}
    {   \setcounter{numberlistc}{0}
        \begin{list}{\arabic{numberlistc}.}
        {\usecounter{numberlistc}
        \setlength{\parsep}{0pt}
        \setlength{\topsep}{3pt}
        \setlength{\itemsep}{0pt}}
        }{ \end{list} }
\newcounter{itemlistc}
\newcounter{enumlistc}
%\renewcommand{\theromanlistc}{(\roman{romanlistc})} %for ref use
%\renewcommand{\thealphlistc}{(\alph{alphlistc})}    %for ref use
\newenvironment{itemlist}
    {   \setcounter{itemlistc}{0}
    \begin{list}{$\bullet$}
        {\usecounter{itemlistc}
        \setlength{\parsep}{0pt}
        \setlength{\topsep}{3pt}
        \setlength{\itemsep}{0pt}}
        }{ \end{list} } 

\newcommand{\rev}[1]{\textcolor{red}{#1}}

\renewcommand{\baselinestretch}{1}

% \newenvironment{packedenum}{
% \begin{enumerate}
% \setlength{\itemsep}{1pt}
% \setlength{\parskip}{0pt}
% \setlength{\parsep}{0pt}
% }{\end{enumerate}}

%\usepackage{setspace}
%\DeclareMathOperator{\rank}{rank}


%\renewcommand{\baselinestretch}{0.895}
%\IEEEoverridecommandlockouts
%\usepackage[font=small, labelfont=bf]{caption}
%\usepackage{caption}


%\renewcommand{\algorithmicrequire}{\textbf{Input:}}
%\renewcommand{\algorithmicensure}{\textbf{Output:}}
%\renewcommand{\algorithmiccomment}[1]{~~~~\textcolor{BrickRed}{// \textit{#1}}}


\begin{document}
\title{ EM-Aware Power Grid IR Drop Fixing Accelerated by Variational AutoEncoder}

%\author{\indent Yibo Liu~\IEEEmembership{Student Member,~IEEE}, 
%Sheldon X.-D. Tan~\IEEEmembership{Senior Member,~IEEE} 
%~\thanks{\indent Yibo Liu, and Sheldon X.-D. Tan are with the Department of Electrical and 
%Computer Engineering, University of California, Riverside, Riverside, CA 92521, USA~(e-mail: stan@ece.ucr.edu).}}


\maketitle
  
\begin{abstract}
  Electromigration (EM) remains the top failure mechanism for
  copper-interconnected deep sub-micro technologies. This paper
  presents a full-chip EM-aware IR drop fixing method for on-chip power
  grid networks Accelerated by Variational AutoEncoder. We proposed a
  Variational AutoEncoder (VAE) accelerated power grid IR drop fixing framework, in which the
  sensitivity computation for the sequence of linear programming (SLP)
  is accelerated by VAE-based machine learning method instead of traditional
  numerical methods.


  We also show that the proposed method can lead up to 90X
  (at least one order of magnitude) speedup over the conventional
  SLP-based method on synthesized power grid benchmarks
  from ARM Cortex-M0 processor designs. Furthermore, compared to the
  recently proposed generative adversarial network (GAN) accelerated conjugate gradient-based
  optimization method, the proposed method can still lead to 2-8X
  speedup on the synthesized power grid networks with higher accuracy in terms of 40$\%$ RMSE reduction in mV.




\end{abstract}
 
\input intro.tex

\input related.tex

\input strategy.tex

\input results.tex

\section{Conclusion}
\label{sec:conclusion}
In this paper, we proposed a VAE accelerated power grid
fixing method based on a recently proposed linear programming based
optimization framework.  But instead of using the numerical matrix
solving method for obtaining the sensitivity information, which is
computationally slow and does not scale well for more violations, we
proposed to use the VAE-based model to accelerate sensitivity
computation.  Numerical results on a number of synthesized power grid
benchmarks from ARM Cortex-M0 processor designs show that the proposed
method can lead to at least one order of magnitude speedup over the
existing SLP based method. Compared to recently proposed
DNN-accelerated conjugate gradient based optimization method the
proposed method can lead to up to 8X speedup on the synthesized power
grid networks.





% \clearpage 
\bibliographystyle{IEEEtran}
\bibliography{../../bib/simulation,../../bib/modeling,../../bib/reduction,../../bib/misc,../../bib/mscad_pub,../../bib/reliability,../../bib/interconnect,../../bib/thermal_power,../../bib/machine_learning,../../bib/physical,../../bib/neural_network}
%\bibliography{bibfile}
%\input change_note.tex
% \input {bio.tex}

\end{document}
