\documentclass[journal]{IEEEtran}
% \usepackage{booktabs} % For formal tables
% \usepackage{times}
% \renewcommand{\rmdefault}{ptm}
\usepackage{multirow}
% \usepackage{array}
\usepackage{amsmath}
\usepackage{subfigure}
% \usepackage{bm}
% \usepackage{amsfonts}
\usepackage{epsfig}
\usepackage{psfrag}
% \usepackage{epstopdf}
%\usepackage{cite}
%\usepackage{url}
\newcounter{numberlistc}
\usepackage{algorithm}
\usepackage{algorithmic}
\usepackage{graphicx}
\usepackage{amssymb}
\usepackage{amsthm}
\usepackage{subfigure}
\usepackage{color}
\usepackage{spverbatim}
\usepackage{setspace}
\usepackage{caption}
%\usepackage{subcaption}

\newenvironment{numberlist}
    {   \setcounter{numberlistc}{0}
        \begin{list}{\arabic{numberlistc}.}
        {\usecounter{numberlistc}
        \setlength{\parsep}{0pt}
        \setlength{\topsep}{3pt}
        \setlength{\itemsep}{0pt}}
        }{ \end{list} }
\newcounter{itemlistc}
\newcounter{enumlistc}
%\renewcommand{\theromanlistc}{(\roman{romanlistc})} %for ref use
%\renewcommand{\thealphlistc}{(\alph{alphlistc})}    %for ref use
\newenvironment{itemlist}
    {   \setcounter{itemlistc}{0}
    \begin{list}{$\bullet$}
        {\usecounter{itemlistc}
        \setlength{\parsep}{0pt}
        \setlength{\topsep}{3pt}
        \setlength{\itemsep}{0pt}}
        }{ \end{list} } 

\newcommand{\rev}[1]{\textcolor{red}{#1}}

\renewcommand{\baselinestretch}{1}

% \newenvironment{packedenum}{
% \begin{enumerate}
% \setlength{\itemsep}{1pt}
% \setlength{\parskip}{0pt}
% \setlength{\parsep}{0pt}
% }{\end{enumerate}}

%\usepackage{setspace}
%\DeclareMathOperator{\rank}{rank}


%\renewcommand{\baselinestretch}{0.895}
%\IEEEoverridecommandlockouts
%\usepackage[font=small, labelfont=bf]{caption}
%\usepackage{caption}


%\renewcommand{\algorithmicrequire}{\textbf{Input:}}
%\renewcommand{\algorithmicensure}{\textbf{Output:}}
%\renewcommand{\algorithmiccomment}[1]{~~~~\textcolor{BrickRed}{// \textit{#1}}}


\begin{document}
\title{Fast Power Grid EM-Aware IR Drop Prediction and Fixing
  Accelerated by Variational AutoEncoder}

%\author{\indent Yibo Liu~\IEEEmembership{Student Member,~IEEE}, 
%Sheldon X.-D. Tan~\IEEEmembership{Senior Member,~IEEE} 
%~\thanks{\indent Yibo Liu, and Sheldon X.-D. Tan are with the Department of Electrical and 
%Computer Engineering, University of California, Riverside, Riverside, CA 92521, USA~(e-mail: stan@ece.ucr.edu).}}


\maketitle
  
\begin{abstract}
  Electromigration (EM) still remains the top failure mechanism for
  copper-interconnected current and future nanometer chip
  technologies.  In this paper, we presents a new EM-aware full-chip
  power grid optimization framework to size the on-chip power grid
  networks to ensure their EM life time during the design. One
  critical issue of the optimization is to compute the sensivity or
  gradients of the cost funciotns with respect to the wire
  geometry. Existing approaches suffers high simulation costs of
  full-chip EM-aware IR drop analysis for such computation. To
  mitigate this issue, we propose to use variational autoencoder (VAE)
  to first model the EM-aware IR drops of a power grid networks, then
  we leverage the auto differential feature of the deep neural
  networks to compute the sensivity information very efficiently. Our
  results show that VAE is better than recently proposed recently
  proposed generative adversarial network (GAN)-based with 40$\%$ RMSE
  reduction. Furthremore, 
  % presents a Variational AutoEncoder (VAE)-based EM-aware IR drop
  % prediction model for the full-chip power grid network.  Compared to
  % the recently proposed generative adversarial network (GAN)-based
  % method, our VAE model has a higher accuracy in terms of 40$\%$ RMSE
  % reduction in mV.  We further extend our VAE-based fast EM-aware IR
  % drop prediction model to a fast full-chip power grid IR-drop fixing
  % framework.  We formulate the power grid EM-IR drop fixing as a
  % optimization problem and solve it with sequence of linear
  % programming (SLP) method.  The optimization is accelerated by the
  % fast acquisition of aging-aware IR drop as the VAE model prediction
  % output, and the sensitivity of aging-induced voltage to the initial
  % PGN tree conductance from the model inference auto gradient.
  We show that the proposed VAE-accelerated EM-aware fixing method can
  lead up to 90X (at least one order of magnitude) speedup over the
  conventional SLP-based method on synthesized power grid benchmarks
  from ARM Cortex-M0 processor design.

\end{abstract}
 
\input intro.tex

\input related.tex

\input strategy.tex

\input results.tex

\section{Conclusion}
\label{sec:conclusion}
In this paper, we proposed a VAE accelerated power grid
fixing method based on a recently proposed linear programming based
optimization framework.  
we proposed to use the VAE-based model to accelerate sensitivity
computation.  Numerical results on a number of synthesized power grid
benchmarks from ARM Cortex-M0 processor designs show that the proposed
method can lead to at least one order of magnitude speedup over the
existing SLP based method. 




% \clearpage 
\bibliographystyle{IEEEtran}
\bibliography{../../bib/simulation,../../bib/modeling,../../bib/reduction,../../bib/misc,../../bib/mscad_pub,../../bib/reliability,../../bib/interconnect,../../bib/thermal_power,../../bib/machine_learning,../../bib/physical,../../bib/neural_network}
%\bibliography{bibfile}
%\input change_note.tex
% \input {bio.tex}

\end{document}
