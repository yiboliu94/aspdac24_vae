\documentclass[journal]{IEEEtran}
% \usepackage{booktabs} % For formal tables
% \usepackage{times}
% \renewcommand{\rmdefault}{ptm}
\usepackage{multirow}
% \usepackage{array}
\usepackage{amsmath}
\usepackage{subfigure}
% \usepackage{bm}
% \usepackage{amsfonts}
\usepackage{epsfig}
\usepackage{psfrag}
% \usepackage{epstopdf}
%\usepackage{cite}
%\usepackage{url}
\newcounter{numberlistc}
\usepackage{algorithm}
\usepackage{algorithmic}
\usepackage{graphicx}
\usepackage{amssymb}
\usepackage{amsthm}
\usepackage{subfigure}
\usepackage{color}
\usepackage{spverbatim}
\usepackage{setspace}
\usepackage{caption}
%\usepackage{subcaption}

\newenvironment{numberlist}
    {   \setcounter{numberlistc}{0}
        \begin{list}{\arabic{numberlistc}.}
        {\usecounter{numberlistc}
        \setlength{\parsep}{0pt}
        \setlength{\topsep}{3pt}
        \setlength{\itemsep}{0pt}}
        }{ \end{list} }
\newcounter{itemlistc}
\newcounter{enumlistc}
%\renewcommand{\theromanlistc}{(\roman{romanlistc})} %for ref use
%\renewcommand{\thealphlistc}{(\alph{alphlistc})}    %for ref use
\newenvironment{itemlist}
    {   \setcounter{itemlistc}{0}
    \begin{list}{$\bullet$}
        {\usecounter{itemlistc}
        \setlength{\parsep}{0pt}
        \setlength{\topsep}{3pt}
        \setlength{\itemsep}{0pt}}
        }{ \end{list} } 

\newcommand{\rev}[1]{\textcolor{red}{#1}}

\renewcommand{\baselinestretch}{0.96}

% \newenvironment{packedenum}{
% \begin{enumerate}
% \setlength{\itemsep}{1pt}
% \setlength{\parskip}{0pt}
% \setlength{\parsep}{0pt}
% }{\end{enumerate}}

%\usepackage{setspace}
%\DeclareMathOperator{\rank}{rank}


%\renewcommand{\baselinestretch}{0.895}
%\IEEEoverridecommandlockouts
%\usepackage[font=small, labelfont=bf]{caption}
%\usepackage{caption}


%\renewcommand{\algorithmicrequire}{\textbf{Input:}}
%\renewcommand{\algorithmicensure}{\textbf{Output:}}
%\renewcommand{\algorithmiccomment}[1]{~~~~\textcolor{BrickRed}{// \textit{#1}}}


\begin{document}
\title{GridVAE: Fast Power Grid EM-Aware IR Drop Prediction and Fixing
  Accelerated by Variational AutoEncoder}

%\author{\indent Yibo Liu~\IEEEmembership{Student Member,~IEEE}, 
%Sheldon X.-D. Tan~\IEEEmembership{Senior Member,~IEEE} 
%~\thanks{\indent Yibo Liu, and Sheldon X.-D. Tan are with the Department of Electrical and 
%Computer Engineering, University of California, Riverside, Riverside, CA 92521, USA~(e-mail: stan@ece.ucr.edu).}}


\maketitle
  
\begin{abstract}
  % Electromigration (EM) is the top failure mechanism for copper-based
  % interconnected for today's and future nanometer chip technologies.
  % The on-chip power grids have to be properly design to ensure the EM
  % life time. Most of the existing approach is based on the sensitivity
  % based optimizations and one critical issue is the high computational
  % costs of full-chip EM-aware IR drop analysis to compute the
  % sensitivity of the fixing target with respect to the wire
  % geometry. In this paper, we propose a fast full-chip power grid
  % EM-aware IR drop prediction and fixing framework. First, we adopt
  % the variational autoencoder (VAE)-based model, called {\it GridVAE},
  % to predict the EM-aware IR drops, which are more accurate and thus
  % training data-efficient than existing generative adversarial network
  % (GAN) based approaches. On top of this, we apply a sequence of
  % linear programming based optimization framework to size the wires by
  % leveraging the auto differential feature of the deep neural networks
  % to compute the sensitivity information very efficiently.  The
  % numerical results show that VAE is better than recently proposed
  % recently proposed GAN-based model with 40$\%$ RMSE reduction.
  % Further on the power grid EM-aware IR drop fixing, we show the
  % proposed VAE-accelerated method can lead up to 80X (at least one
  % order of magnitude) speedup over the conventional SLP-based method
  % on synthesized power grid benchmarks from ARM Cortex-M0 processor
  % design.


  Electromigration (EM) remains the primary failure mechanism for
  copper-based interconnects in today's and future nanometer chip
  technologies. To ensure the longevity of on-chip power grids,
  effective EM-aware IR drop analysis is crucial. However, the
  existing sensitivity-based optimization approaches suffer from high
  computational costs, particularly for full-chip analysis.  In this
  paper, we propose a novel and efficient framework for EM-aware IR
  drop prediction and fixing in full-chip power grids. Our approach
  leverages a variational autoencoder (VAE)-based model, known as {\it
    GridVAE}, for accurate EM-aware IR drop predictions. Compared to
  traditional generative adversarial network (GAN) based methods, our
  VAE-based model offers improved accuracy and data efficiency.
  Building on the accurate predictions of GridVAE, we appliy sequence
  of linear programming-based optimizations to efficiently size the
  wires. This optimization framework takes advantage of the
  auto-differential capabilities of deep neural networks,
  significantly reducing computation time for sensitivity information
  computation. Our numerical results demonstrate the superiority of
  VAE over recently proposed GAN-based models, showcasing a remarkable
  40\% reduction in RMSE. Moreover, our proposed VAE-accelerated
  method achieves up to an 80X speedup (at least one order of
  magnitude) compared to conventional SLP-based methods for power grid
  EM-aware IR drop fixing. We validate our approach using synthesized
  power grid benchmarks from ARM Cortex-M0 processor design.
\end{abstract}
 
\input intro.tex

\input related.tex

\input strategy.tex

\input results.tex

\section{Conclusion}
\label{sec:conclusion}
In this paper, we proposed a variational autoencoder (VAE) based
full-chip EM-aware IR drop estimation model, called {\it GridVAE}, for
on-chip power grid networks. The new deep neural network models allow
fast sensitivity computations of target cost function with respect to
the wire geometries. This enable accelerated power grid fixing based
on a recently proposed linear programming based optimization
framework.  The numerical results show that {\it GridVAE} is better
than recently proposed recently proposed GAN-based model with 40$\%$
RMSE reduction.  Furthermore, we show the proposed {\it
  GridVAE}-accelerated method can lead up to 80X (at least one order
of magnitude) speedup over the conventional SLP-based method on
synthesized power grid benchmarks from ARM Cortex-M0 processor design.




% \clearpage 
\bibliographystyle{IEEEtran}
\bibliography{../../bib/simulation,../../bib/modeling,../../bib/reduction,../../bib/misc,../../bib/mscad_pub,../../bib/reliability,../../bib/interconnect,../../bib/thermal_power,../../bib/machine_learning,../../bib/physical,../../bib/neural_network}
%\bibliography{bibfile}
%\input change_note.tex
% \input {bio.tex}

\end{document}
