+\section{Related works}
\label{sec:related}

In this section, we first summarize and review some related preliminary works of on-chip power grid EM-induced IR drop analysis and mitigation approaches.

\subsection{Full-chip EM-induced IR drop analysis}
\label{subsec:emspice}
As mentioned in the previous section, EM is a physical phenomenon that can lead to resistance increase or even open-wire segments.  The IR drop of the power grid wires may change due to the EM-induced aging effect. With enough wires nucleate, the EM-induced IR drop eventually leads to timing violations. This means we have to consider the power girds IR drop as time-varying characters ~\cite{SunYu:TDMR'20, Huang:TCAD'15, Chatterjee:2018TCAD,SukharevNajm:2018TDMR}. 
On the other hand, the failed wire segments change the current distributions of all the interconnect wires, which may further accelerate the failure process. Hence, to emulate the on-chip power grid IR-drop after aging effect, one has to consider the interplay between the two physics: electrical characteristics and hydrostatic stress in the interconnect wires.

{\it EMspice}~\cite{SunYu:TDMR'20,EMspiceSourceCode} is an open source tool that conducts the full-chip power grid network coupled EM-IR drop simulation with the dynamic interplay between the hydrostatic stress and electronic current/voltage. It solves the coupled time-varying partial differential equations in the time domain to obtain the stress evolution, and finally reports resulted IR drop and EM failure hotspots at the target aging time, such as 10 years.  The tool consists of a finite difference time domain (FDTD) solver for EM stress and a linear network DC solver for IR drop, which can be described as

%\begin{align}
%	\label{eq:em}
%		&{\bf C} \dot{\sigma}(t)  = {\bf A} \sigma(t) + {\bf P}I(t),  \\
%		&\mathcal{V}_v(t)  = \int_{\Omega_L}\frac{\sigma(t)}{B}d\mathcal{V},  \\ 
%		&{\bf M}(t) \times u(t)  = {\bf P} I(t), \\
%		&{\bf \sigma}(0)  = [\sigma_1(0), \sigma_2(0), ..., \sigma_n(0)] \,\, ,at \,\, t = 0 
%\end{align}

\begin{align}
	\label{eq:em}
	\begin{split}
		&{\bf C} \dot{\sigma}(t)  = {\bf A} \sigma(t) + {\bf P}I(t),  \\
		&\mathcal{V}_v(t)  = \int_{\Omega_L}\frac{\sigma(t)}{B}d\mathcal{V},  \\ 
		&{\bf M}(t) \times u(t)  = {\bf P} I(t), \\
		&{\bf \sigma}(0)  = [\sigma_1(0), \sigma_2(0), ..., \sigma_n(0)] \,\, ,at \,\, t = 0 
	\end{split}
\end{align}


\textcolor{red}{These two paragraphs need merge} In the above equations .~\eqref{eq:em}, ${\bf M}(t)$ is the time varying power grid conductance matrix, as the resistance changes due to the EM failure process. ${{\bf P}}$ is the input matrix and I(t) represents the current sources from the chip. ${{\bf C}}$ is the identity matrix and ${{\bf A}}$ is a coefficient matrix. $\sigma_{n}(0)$ denotes the initial stress at time step $t = 0$, for node $n$. For each new time step, the stress from previous simulation step is used as the initial condition.

The above equations in .~\eqref{eq:em} are coupled and must be solved together. Linear network IR drop solver passes time-dependent current densities and P/G layout information to the finite difference time domain (FDTD) EM solver. The FDTD EM solver provides the IR drop solver with new resistance information.  these two simulations are coupled together. Such iterative coupled analysis on long target lifetime can be extremely time consuming for very large power grid networks ~\cite{SunYu:TDMR'20,EMspiceSourceCode} 






%There are numerical methods to conduct power grid IR drop analysis, such as hierarchical methods, random walk methods, Krylov-subspace methods, multi-grid techniques, and vector-less verification methods.


%PDN design is a complex iterative optimization task that strongly influences the performance, area, and cost of a chip. 
%To reduce the design time 
Recent studies pay attention to ML-based PDN IR drop estimation ~\cite{LinFang:2018vts,Fang:2018dynireco,HoKahng:ICCAD'19,Xie:2020powernet,Sachin:ASPDAC'21}. 
The key ideas of these methods aims to substitute the standard full-chip IR drop analysis tool to data-driven feature extraction and training. Lin {\it et al.}~\cite{LinFang:2018vts} tried to extract power and physical features from cells and layouts to conduct the full-chip dynamic IR drop analysis . Fang {\it et al.}~\cite{Fang:2018dynireco} proposed to training the models for the localized layout region to improve the scalability.  A convolutional neural network (CNN) based model that is able to incorporate design-dependent features during pre-processing is proposed by Xie {\it et al.}~\cite{Xie:2020powernet}, which is transferable across different designs.
Ho {\it et al.}~\cite{HoKahng:ICCAD'19} proposed incremental IR drop prediction and mitigation by applying more electrical and physical features for the gradient boosting framework training. Chhabria {\it et al.}~\cite{Sachin:ASPDAC'21} proposed {\it IREDGe}, which is a CNN-based generative network method to predict on-chip IR drop contours with image-to-image and sequence-to-sequence translation tasks.
Recently Zhou {\it et al.}~\cite{ZhouJin:ICCAD'20} proposed an EM-induced IR drop analysis framework based on conditional GAN. The framework regards the time and selected electrical features as input images and outputs the voltage map. Furthermore, it has been demonstrated that the trained CGAN models can be used to compute the sensitivity of objective function with respect to the wire resistances for localized power grid fixing.  These machine learning methods indeed have achieved significant progress in IR drop estimation. But none of them takes EM aging effects into consideration.
In this work, we further leverage {\it GridNet}-like DNN for full-chip power grid optimization subject to the EM-induced IR drop constraints.


\subsection{Existing EM-aware PDN optimization}
 \label{subsec:exist_pgfix}
 % There have been a number of works proposed recently on wire segment sizing of power grid networks in order to fix the EM failures and IR drop considering the multi-segment interconnect wires. Zhou {\it et al.}~\cite{ZhouSun:ASPDAC'18,ZhouSun:TVLSI'19} proposed a power grid network sizing method based on the multi-segment EM immortality check criteria. It automatically considers all the wire segments and their interactions in an interconnect tree. However, the EM immortality constrained optimization is still conservative as it requires all the interconnect trees to be immortal, i.e., void nucleations are not allowed. Chang {\it et al.}~\cite{ChangBaranwal:ASPDAC'18} proposed a learning-based EM violation waiver system, which investigates every EM violation and takes an expert decision to either ignore the violation (waive-off) or resolve it (must-fix) in the design. However, this system cannot take the violation fix action.
 Besides the full-chip aging-aware IR-drop estimation , one also need to fix or alleviate the excessive IR-drop to ensure a robust PDN design. Many past research apply nonlinear or linear optimization methods~\cite{DuMa:DAC'89,Tan:DAC'99,Wang:TCAD'05,ZhouSun:TVLSI'19, Sukharev:2019pg} to properly size the PDN wires. The goal of these optimization strategies is to meet the IR drop requirement at the target lifetime \textcolor{red}{or extend the MTTF} with minimize metal routing area.
 
 The SLP-based method was proposed first in~\cite{Tan:DAC'99} based on Black's equation. Then this method was extended to consider multi-segment wires ~\cite{ZhouSun:ASPDAC'18}. But this method can be too conservative as it requires all the wires to be immortal after optimization. Although this method has been extended to consider a targeted lifetime by allowing some wires to fail and optimizing the rest of the wires~\cite{ZhouSun:TVLSI'19}.  Recently \cite{Sukharev:2019pg} proposed to directly optimize EM-induced IR drops on the time-varying power grid networks EM caused by EM-induced aging using SLP (called successive linear programming).  The EM-induced aging effects on IR drop are computed by solving Korhonen equations.  Since it models the power grid network as a nonlinear time-varying system, sensitivities of the IR drop with respect to the wires have to be computed via the matrix solving methods at each time step. The circuit matrix construction and solving process are severely time-consuming, especially when the number of violations and size of the power grid is large.

To solve the optimization problem, we need two main inputs: v(T=t) and dv/ds. There are some methods to compute dv/ds, such as adjoint network method. Vxx also proposed a matrix solving method,
The main drawback is, the dv/ds has to be computed line
 



 % This method aims to re-size the individual wires to meet the target IR drop criteria. The approach applies successive linear programming method, which iteratively expand a non-linear problem to a linear space and then optimize it. Each iteration contains the following steps: first linearize the nodes voltage drop around the current \textcolor{blue}{grid topology(wrong term)}, then determine the gradient descent direction at the current solution point through partial differential computation, finally find the updated solution point by solving a linear programming problem. Clearly this numerical computation method takes a huge amount of workload, hence the speed and scalability are the main issues.
 % We hereby propose our DNN-based machine learning method to accelerate this optimization approach. We replace the
