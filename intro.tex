\section{Introduction}
\label{sec:intro}
With the complexity of modern very large scale integration (VLSI) designs, reliability is becoming a more serious concern. Among the reliability issues, electromigration (EM) remains the top failure mechanism for copper-based interconnects in all the sub-nanometer technologies. EM is a physical diffusion phenomenon that the metal atoms migrate along the direction of several driving forces, such as the applied electrical field.  During migration, when the hydrostatic stress inside the embedded metal wire reaches the critical level, a void and a hillock will be formed by conducting electrons at the cathode and anode.
The EM crisis becomes even worse as technology advances to smaller manufacturing processes. The International Roadmap for Devices and Systems (IRDS) \cite{IRDS20} predicts that the allowable current density will continue to decrease due to EM, while the required current density to drive the gates will continue to increase. Developing and employing more accurate EM models and less conservative EM sign-off and assessment techniques for EM-aware designs and runtime management is critical. 

Power distribution networks (PDN) provide power to transistors from top metals on a chip. They have a direct impact on chip performance and reliability. But PDNs are usually vulnerable to the EM-induced failures, as the current densities on the PDN are large and unidirectional. In order to design robust PDN, designers have to properly size the wires to meet the area and IR drop requirement. Many research works have been investigated in the past based on nonlinear or sequence of linear programming (SLP) methods~\cite{ChBr:TCAD'88,DuMa:DAC'89,Tan:DAC'99,Wang:TCAD'05,ZhouSun:TVLSI'19, Sukharev:2019pg, ZhouYu:ASPDAC'20,ZhouJin:ICCAD'20}.
Zhou {\it et al.}~\cite{ZhouSun:TVLSI'19,ZhouChen:Integration'21} proposed a power grid network sizing method based on a multi-segment EM immortality check criteria. It automatically considers all the wire segments and their interactions within an interconnect tree. However, the EM immortality-constrained optimization is too conservative as it requires all the interconnect trees to be immortal, i.e., void growth is not allowed. To further mitigate this issue, Moudallal {\it et al.}~\cite{Sukharev:2019pg} proposed to directly consider EM-induced IR drops instead of EM constraints on the time-varying power grid networks. It can consider post-voiding resistance change of wires based on finite difference analysis of EM-induced stress in multi-segment wires, and the resulting nonlinear problem is solved by applying successive linear programming. This method, however, may suffer high computational costs as the sensitivities of those violating nodes need to be computed by solving the circuit matrices. This computation workload of matrices construction and matrices solving increases polynomially as the size of the power grid increases. Also, the scalability of SLP-based method is quite limited when the number of power grid violations increase, since the optimization flow has to solve more sensitivity matrices.
   
In ~\cite{ZhouJin:ICCAD'20} Zhou {\it et al.} proposed a localized EM-aware IR drop fix for power grid networks. The optimization is carried out based on the sensitivities method in which the sensitivities were computed by the CGAN-based deep neural network (DNN) modeling method called {\it GridNet}. This method was initially only applied to localized optimization, and it has been extended to full-chip optimization~\cite{HanLiu:TCAD'22-23}.  The method shows the promise of leveraging DNNs in the field of power grid optimization. DNN-based sensitivities computation is much faster due to the efficient backpropagation scheme and scales well when the number of failed trees in the power grid increase.

However they are some drawbacks of GAN-based strategy, such as GANs are challenging to train due to the need to optimize opposing objectives of the generator and discriminator, leading to issues like mode collapse. Recently,  (VAE) gained much traction as it has the training and better interpretability of latent space, compare to the contemporaries, such as GAN, which provides them with a robust mathematical grounding and better generated result in some scenarios. 

Based on those observations, we propose a fast EM-aware full-chip power grid IR drop fixing method by leveraging the modeling power of VAE. Our key contributions are decomposed as follows:

\begin{itemlist}
	
\item First we propose a VAE-based framework to fast predict the EM-aware full-chip power grid IR drop. Our new model reduced approximately 40$\%$ RMSE in mV, Compared to the GAN-based model.

\item Second, instead of using the circuit matrix solving-based sensitivity computation method in~\cite{Sukharev:2019pg}, we quickly acquired the sensitivity information from the network auto-gradient,  which is similar to the CGAN in ~\cite{ZhouJin:ICCAD'20}, to compute the EM-induced IR drop at the given lifetime and the sensitivities of the IR drops with respect to the wire segment width.  Our VAE-based strategy significantly improves the running speed of the current SLP-based voltage failure fixing methods.
%We propose using a conditional generative adversarial network(CGAN)-based tool {\it GridNet}~\cite{ZhouJin:ICCAD'20} to quickly acquire the sensitivity matrix for the power grid fixing, avoiding a heavy-loaded numerical computation process and improving the running speed.

\item We show that the VAE-based IR drop modeling techniques scale well for full-chip IR drop and sensitivities analysis even with a large number of violations and thereby improving the efficiency of the whole optimization process.
  
  % Ameliorate the scalability in the case of a large amount of power grid tree failures. The traditional fixing method's time consumption jumps significantly as the number of power grid trees that need modification increases. The time consumption of the proposed {\it GridNet}-based fixing approach does not change significantly due to the amount increase of the modified trees in a given power grid.
	

\item Experimental results on a number of synthesized power grid
  benchmarks from ARM Cortex-M0 processor designs indeed show that
  the Joule heating aware EM optimization indeed leads to
  less area overheads to fix the power grids compared to the
  optimization without considering Joule heating first. This means
  that existing EM aware IR drop analysis without considering Joule
  heating effects will overestimate the overhead for fixing
  the EM-induced IR drops. Second, the proposed method can
  provide up to 90X (at least one order of magnitude) speedup over the
  existing SLP-based method.
  Furthermore, compared to the recently
  proposed VAE-accelerated conjugate gradient-based optimization method,
  the proposed method can lead to 2-8X speedup on the synthesized
  power grid networks.
  
	
\end{itemlist}

The rest of the paper is organized as follows:
Section~\ref{sec:related} reviews the related preliminary works on the
EM-induced IR drop analysis and current EM-aware power grid
optimization strategy. Section~\ref{sec:strategy} introduces the
DNN-based EM-aware IR drop prediction method considering the Joule
heating effects. Experiment setup, numerical results, as long as
analysis and discussions are summarized in Section~\ref{sec:results}.
Section~\ref{sec:conclusion} concludes the paper.
